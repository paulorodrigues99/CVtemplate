%!TEX TS-program = xelatex
\documentclass[]{friggeri-cv}
\addbibresource{bibliography.bib}

\begin{document}
\header{Paulo}{Rodrigues}
       {Programador}


% In the aside, each new line forces a line break
\begin{aside}
  \section{Sobre}
    Rua buique
    245, Piedade
    Pernambuco - Brazil
    ~
    \href{mailto:pvrg.ava@gmail.com}{pvrg.ava@gmail.com}
    \href{https://github.com/paulorodrigues99}{GitHub}
    \href{http://wa.me/5581995261010}{WhatsApp}
  \section{Idiomas}
    portugues - fluente
    espanhol \& inglês - intermediário
  \section{programming}
    {\color{red} $\varheartsuit$} JavaScript
    (node.js)
    CSS \& HTML5
\end{aside}

\section{Profissional}

Profissional com atuação voltada ao Desenvolvimento de Software, formado em Gestão de Tecnologia da Informação. Participei de projetos de desenvolvimento de software e RPA, sugeri arquiteturas para problemas utilizando RPA.
Desenvolvi soluções utilizando as seguintes tecnologias: C#, Javascript, NodeJs, implementando Entity Framework, ReactJs e Express.

Desenvolvimento voltado a banco de dados com: SQL Server e MongoDb

Conhecimento da cultura de desenvolvimento ágil, utilizando as principais metodologias ágeis para gestão e planejamento dos projetos, através de ferramentas como TFS, Trello, SVN e Git. 

\section{Educação}

\begin{entrylist}
  \entry
    {2019}
    {Tecnologo {\normalfont em gestão de T.I}}
    {Faculdade dos Guararapes}
    {\emph{Transferi de BSI (Sistemas de Informação) na UPE para este curso em 2017.}}
\end{entrylist}

\section{Experiencias}

\begin{entrylist}
  \entry
    {02–08 2019}
    {Avanade, Recife}
    {ASE - Associate software engineer.}
    {\emph{Atuação com Csharp e desenvolvimento de soluções fim-a-fim com automação e frameworks de automação de processos como BluePrism | Automation Anywhere | AutoIt.}}
\end{entrylist}

\section{Iniciativas}

\begin{entrylist}
  \entry
    {2019}
    {Projeto Hermes}
    {{Desenvolvedor}}
    {Inovamos em segurança nas escolas e estamos em constante desenvolvimento de ideias para escolas }
    \entry
    {2017}
    {Cidade que queremos}
    {\href{http://www.portosocial.com.br/incubados/}{Lista de Incubados no Porto Social}}
    {Atuamos na construção de infraestrutura das cidades da Região Metropolitana do Recife na área de paisagismo, lazer, esporte e educação.}
\end{entrylist}

\end{document}
